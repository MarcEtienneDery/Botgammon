\documentclass{article}

\usepackage[utf8]{inputenc}
\usepackage[top=1.5in, bottom=1.5in, left=1in, right=1in, headsep=0.5in]{geometry}
\usepackage[T1]{fontenc}
\usepackage[frenchb]{babel}
\usepackage{array}
\usepackage{fancyhdr}
\usepackage{amssymb}
\usepackage{mathtools}
\usepackage[final]{pdfpages}
\usepackage{tabulary}

\usepackage{pifont}
\newcommand{\cmark}{\ding{51}}
\newcommand{\xmark}{\ding{55}}

\pagestyle{fancy}
\rhead{O4L - BotGammon}
\lhead{INF4230 - Groupe 10}

% pour éviter d'avoir à faire des \noindent partout!

\title{
\huge{Annexe}\\
\vfill
}

\begin{document}

\maketitle

\thispagestyle{empty}
\clearpage

\openup .5em

\section{Tableau des résultats}

{
\begin{center}
    \renewcommand{\arraystretch}{1.6}
    \begin{tabulary}{1.0\textwidth}{| L || C | C | C | C | C |}
        \hline
        \textbf{Fonction d'évaluation} & \textbf{Élagage alpha-bêta} & \textbf{Pourcentage de victoire} & \textbf{Nombre
    de parties} & \textbf{Profondeur} & \textbf{Temps moyen des coups}\\
        \hline
        Aucune & - & 0.5\% & 1000 & - & < 1ms\\
        Heuristique Franklin & \cmark & 30,0\% & 1000 & 1 & 2ms\\
        Heuristique Franklin & \xmark & 27,8\% & 1000 & 1 & 2,5ms\\
        Heuristique Simple & \cmark & 38,8\% & 1000 & 1 & 2ms\\
        Heuristique Simple & \xmark & 39,1\% & 1000 & 1 & 2ms\\
        Heuristique Franklin & \cmark & 30,0\% & 10 & 2 & 3,1s\\
        Heuristique Franklin & \xmark & 0,0\% & 10 & 2 & 1,7s\\
        Heuristique Simple & \cmark & 40,0\% & 10 & 2 & 1,2s\\
        Heuristique Simple & \xmark & 0,0\% & 10 & 2 & 2,1s\\
        Heuristique Simple & \cmark & 30,0\% & 10 & Profondeur itérative & 1s (constant)\\
        Heuristique Simple & \xmark & 0,0\% & 10 & Profondeur itérative & 1s (constant)\\
        \hline
    \end{tabulary}
\end{center}
}

\end{document}
