\documentclass{article}

\usepackage[utf8]{inputenc}
\usepackage[top=1.5in, bottom=1.5in, left=1in, right=1in, headsep=0.5in]{geometry}
\usepackage[T1]{fontenc}
\usepackage[frenchb]{babel}
\usepackage{array}
\usepackage{fancyhdr}
\usepackage{amssymb}
\usepackage{mathtools}
\usepackage[final]{pdfpages}

\pagestyle{fancy}
\rhead{O4L - BotGammon}
\lhead{INF4230 - Groupe 10}

% pour éviter d'avoir à faire des \noindent partout!

\title{
\Large{Université du Québec à Montréal}\\
\vspace{3cm}
\huge{BotGammon: Joueur artificiel de backgammon}\\
\vspace{3cm}
\Large{Travail présenté à \\M. Éric Beaudry} \\
\vspace{1.5cm}
\Large{Dans le cadre du cours \\INF4230-10 – Intelligence Artificielle} \\
\vspace{1cm}
\author{Marc-Étienne Déry DERM06099201\\Dominick Latreille LATD19099100\\François Bilodeau BILF31089104\\Philippe Pépos Petitclerc PEPP03049109\\Équipe O4L}
\date{\vspace{0.5cm} 15 décembre 2014}
\vfill
}

\begin{document}
\maketitle

\thispagestyle{empty}
\clearpage

\openup .5em

\section{Introduction}
L’application que nous présentons est un joueur artificiel de backgammon.
Plusieurs techniques d’intelligence artificielle ont été développées pour
résoudre ce genre de jeu, et de nouvelles techniques apparaissent encore dans
les temps récents au fur et à mesure que diverses techniques d’IA évoluent. Nous
utilisons le logiciel libre GNU Backgammon afin d’avoir une interface nous
permettant de tester notre application contre différents niveaux d’intelligence
artificielle. Ce logiciel est très complet et comprend une panoplie de fonctions
nous permettant de tester différentes situations.

L’idée d’un joueur artificiel pour le jeu de backgammon nous a semblé
intéressante tant au niveau du processus de création qu’au niveau du résultat
final. La présence du hasard, associée à un bon niveau de stratégie rend ce jeu
de table original et stimulant. Le joueur artificiel permet à un joueur apprenti
de se pratiquer et d’évaluer ses capacités, en fonction du niveau de
l’intelligence de l’agent.

\section{Problématique}
Le type de problème à résoudre lorsqu’on développe un joueur artificiel de
backgammon est un problème de prise de décision avec probabilité. Il s’agit
d’évaluer le meilleur coup dans la limite des coups possibles en tenant compte
des probabilités des lancers de dés. 

Le backgammon est un environnement entièrement observable, puisqu’on voit le
plateau de jeu en tout temps. Le joueur artificiel fait face à un environnement
stochastique puisqu’il doit brasser deux dés avant de pouvoir jouer. Il ne sait
donc jamais avec certitude la distance que ses pions parcourront, mais il a
accès aux probabilités des différents lancers de dés possibles. Ensuite,
l’environnement est multiagent, car le backgammon se joue à deux joueurs. Il est
également séquentiel, puisque chaque décision a un effet sur les décisions
futures des deux joueurs. De plus, le backgammon est un environnement statique,
car aucun changement à l’environnement ne peut être effectué par un agent
externe et parce qu’il n’y a pas d’horloge, contrairement aux échecs dans
certains cas. Pour finir, le  joueur artificiel fait face à un environnement
discret, puisque les endroits où l’on peut déposer les pions sont bien définis
et que chaque coup joué est distinct d’un autre.

Le défi dans la conception d’un joueur artificiel de backgammon réside dans le
fait que le nombre de positions est très élevé, alors la solution naïve qui
consiste à vérifier le meilleur coup dans une table de positions n’est pas
réalistement faisable. Pour calculer le nombre de positions légales possibles
sur le plateau de jeu, posons:


\begin{equation}
    C(n,m) = n!m!(n-m)!
\end{equation}
représentant le nombre de façons de sélectionner m pions dans un ensemble de n
pions et :
\begin{equation}
    D(n,m) = C(n+m-1,m)
\end{equation}
représentant le nombre de façons de distribuer m pions de la même couleur sur n
\emph{flèches}\\

En supposant que les pions noirs occupent m flèches sur les 24 possibles, il y a
donc $ C(24,m) $ façons de sélectionner les flèches, pour m pions. Le reste des
$ 15-m $ pions peuvent être sur les mêmes flèches que les m précédents, sur le bar ou à
l’extérieur du plateau; le moyen de les distribuer est alors $ D(m+2,15-m) $. Les
pions noirs ont $ 26-m $ endroits disponibles, alors ils peuvent être distribués en
$ D(26-m,15) $ façons. Le nombre total de positions possibles est donc:
\begin{equation}
    m = \sum_{m=0}^{15}C(24,m) \times D(m+2, 15-m) \times D(26-m,15)\\
\end{equation}
\begin{equation}
    m = 18 528 584 051 601 162 496
\end{equation}

Ajoutons à cela le hasard qu’amènent les dés, le backgammon est loin d’être
trivial à jouer pour un ordinateur.

\section{Pertinence de la technique d'intelligence artificielle}
Parmi les multiples techniques d’intelligence artificielle existantes, celles
qui sont les plus utilisées pour le cas du jeu de Backgammon sont la stratégie
de prise de décisions expectiminimax et l’apprentissage par réseaux de neurones
(comme c’est le cas pour GNU Backgammon). Ces approches sont très intéressantes,
mais nous avons décidé d’opter pour l’algorithme expectiminimax avec élagage
alpha-bêta et profondeur itérative pour notre IA puisqu’il fonctionne avec les
jeux à somme nulle comme le backgammon, en plus d’intégrer des éléments de
hasard. C’est aussi une question de temps; nous désirions obtenir un bon
résultat le plus rapidement possible, sans avoir à laisser un algorithme
d’apprentissage tourner pendant des heures ou des jours. L’heuristique au
backgammon peut être sans cesse améliorée, et c’est donc ce sur quoi nous nous
sommes penchés. 

L’algorithme expectiminimax est une variation de minimax qui possède en plus du
noeud \emph{min} et \emph{max} un noeud \emph{chance}. Ce noeud chance représente l’élément de
probabilité du problème associé, qui est dans notre cas les deux dés lancés
avant chaque coup. Tout comme le minimax, cet algorithme va créer un arbre
récursif et alterner entre le joueur \emph{min} pour qui l’on cherche à minimiser son
gain et le joueur \emph{max} que l’on veut maximiser. Par contre, le noeud chance sera
appliqué avant chaque position hypothétique du plateau dans l’arbre, donc avant
le joueur \emph{min} et avant le joueur \emph{max}.  Contrairement aux noeuds \emph{min} et
\emph{max} qui prennent les valeurs d’utilité de leurs enfants, le noeud de chance
prend comme valeur la probabilité d’avoir une combinaison de dés quelconques
multipliée par la valeur du noeud \emph{min} ou \emph{max}. Pour le backgammon, nous avons
une probabilité de 1/18 d’avoir deux dés différents ou 1/36 d’avoir un double.
La complexité temporelle d’expectiminimax est $ O(n^db^d) $ plutôt que $ O(b^d) $ pour
minimax, où n équivaut au nombre de possibilités des résultats de chance. Dans
le cas du backgammon, il y a 21 combinaisons de dés possibles.

Pour la profondeur itérative, c’est une méthode qui nous permet de faire une
recherche le plus profond possible avec le temps alloué pour l’IA. Dans le cas
où l’on ne réussit pas à terminer une itération à une certaine profondeur par
manque de temps, on prend le résultat de la précédente itération. Nous utilisons
également l’élagage alpha-bêta pour notre expectiminimax afin de réduire le
nombre de noeuds à visiter. Cela permet donc de couper certaines branches dans
l’arbre de recherche lorsque l'on voit qu'il est impossible qu'un joueur le
sélectionne.

Afin de rendre la tâche possible, nous avons  pris en considération certaines
hypothèses. Tout d’abord, nous considérons qu’il faut 1 point seulement pour
gagner (1 partie = 1 point), contrairement aux parties de tournoi qui demandent
habituellement entre 3 et 7 points pour gagner. Cela nous facilite la tâche pour
les tests et est tout de même un bon indicateur pour la prise de décisions. De
plus, nous n’utilisons pas le dé doubleur. Le fait de ne pas pouvoir doubler
nous permet d’avoir un gain en rapidité pour chaque coup puisque nous n’avons
pas besoin de calculer la probabilité de gagner avant de brasser les dés.

\section{Résultats}
Afin d’évaluer l'algorithme implémenté, nous avons effectué un grand nombre de
parties contre l’algorithme de base de GNU Backgammon. Nous avons testé avec
deux heuristiques différentes, avec élagage alpha-bêta des noeuds \emph{min} et \emph{max} et
à différentes profondeurs. Nous avons aussi fait quelques tests avec une
profondeur itérative. La première heuristique que nous avons réalisée, surnommée
Franklin, pénalise les pions non protégés, récompense les pairs et les groupes
de pairs (qui bloquent le passage), récompense lorsqu’il mange un ennemi, et
récompense lorsqu’il sort un pion du plateau de jeu. Le tout est pondéré selon
la position des pions. Pour la deuxième heuristique, plus simple, nous
pénalisons les pions non protégés, récompensons les pions protégés, et
récompensons lorsqu’il y a des pions ennemis dans son bar. Certains résultats
que nous avons obtenus sont disponibles en annexe.

\section{Conclusion}
Comme on peut le constater, notre algorithme, s’il est effectué à profondeur 1,
est très rapide. Il joue environ une partie à la seconde, avec un taux de
victoire d’environ 30\% pour l’heuristique Franklin, et 39\% pour l’heuristique
Simple. Quant à la profondeur 2, le temps de calcul est augmenté de façon
exponentielle, avec plus d’une seconde par coup (environ 3 minutes par partie).
Nous avons donc testé avec moins de parties par manque de temps. La différence
de résultats entre la profondeur 1 et 2 est malheureusement minime et donc peu
concluante. Lorsque l’on a essayé de faire des tests avec la profondeur
itérative, nous n’avons remarqué aucune amélioration. On note même une
diminution du taux de victoire (même si l’échantillon est faible) lorsque nous
atteignions des profondeurs supérieures à 2 et beaucoup plus. Ce problème est
sûrement dû à une heuristique peu efficace dans les noeuds plus profonds ou à un
défaut dans l’expectiminimax.

Parmi les pistes de solution, il est très probable que nous n’ayons pas réussi à
trouver les deux constantes qui permettent d’évaluer adéquatement les feuilles
en considérant les noeuds chances. La fonction d’évaluation des noeuds chances
devrait seulement modifier le poids relatif des meilleures actions et pas leur
ordre.

\subsection{Identification de pistes pour l’amélioration de Botgammon}
\begin{itemize}
    \item Simulation de Monte-Carlo:\\
        À chaque position, faire jouer l’algorithme contre lui-même des milliers de fois
        en utilisant des dés aléatoires. En évaluant le pourcentage de victoire, cela
        peut pratiquement remplacer l’heuristique.
    \item Il serait même possible de construire une base de données pour toute les
        positions rencontrées et leur chance de victoire. Nous pourrions ainsi
        facilement inclure le dé doubleur ainsi que la possibilité d’abandonner une
        partie si on est sûr de perdre. Si la position est dans la base de donnée, on
        vérifie, sinon on fait une simulation de Monte-Carlo.
    \item Élagage pour les noeuds chance
    \item Meilleure heuristique
\end{itemize}


\section{Répartition des tâches}
\begin{itemize}
    \item \textbf{Dominick}: Document, présentation, fonctions pour la grille de jeu, expectiminimax,
        tests et résultats
    \item \textbf{François}: Heuristique, liste des coups possibles, mouvement de pion, document,
        tests et résultats, présentation
    \item \textbf{Marc-Étienne}: Expectiminimax, liste des coups possibles, structure du programme,
        interaction avec GNU Backgammon, tests et résultats, heuristique
    \item \textbf{Philippe}: Document, test et résultat, généricisation du code, expectiminimax
        sans alpha-bêta, profondeur itérative avec et sans alpha-bêta
\end{itemize}

\end{document}
